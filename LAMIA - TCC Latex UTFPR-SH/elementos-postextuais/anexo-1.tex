\annexchapter{A}{EXEMPLOS DE OBJETIVO GERAL E ESPECÍFICOS }


Exemplo 2 \\



\begin{mdframed}
	
1.1 OBJETIVO GERAL

O objetivo deste trabalho é modelar, analisar e simular processos de workflow utilizando redes de Petri contínuas com base na ferramenta MATLAB Petri net Toolbox. 

1.1.1 Objetivos Específicos

1)	Mostrar os procedimentos para produzir modelos baseados em redes de Petri contínuas para representar processos de workflow; 

2)	Elaborar um estudo comparativo entre os resultados das simulações do modelo discreto e do modelo contínuo. \linebreak

\end{mdframed}


Exemplo 3  \\



\begin{mdframed}
	
1.1 OBJETIVO GERAL

Aplicar ferramentas otimizadas de visão computacional e aprendizado de máquina supervisionado para verificar o desempenho de algoritmos computacionais de reconhecimento facial e indicar sua viabilidade no desenvolvimento de ferramentas de controle de acesso de indivíduos, a fim de avaliar a robustez de cada algoritmo, usando um conjunto de 3 bancos de dados: Yale, ORL e outro criado com a junção dos dois bancos anteriormente citados.

1.1.1 Objetivos Específicos
\begin{enumerate}
	\item Levantar as principais abordagens de reconhecimento facial e estratégias de cada algoritmo; 
	
	\item Identificar as contribuições e limitações de cada método em diferentes ambientes e tipos de imagem, classificando os algoritmos e técnicas por eficiência e uso;
	
	\item Elaborar um estudo comparativo dos algoritmos com diferentes bancos de imagens, analisando os aspectos intrínsecos da imagem na distorção de cada método;
	
	\item Elaborar uma rotina de experimentos e visualizar os resultados com separação de algoritmos e bases para justificar a indicação de uso de cada método pelo tipo de imagem especializada.\linebreak
	
\end{enumerate}

\end{mdframed}

